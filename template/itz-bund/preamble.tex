% Switch to german spelling
\usepackage[ngerman]{babel}
% Sets the correct encoding and font
\usepackage[explicit]{titlesec}
\usepackage{titling}


% Use color for fonts etc.
\usepackage{xcolor}

% Main color of ITZ-Bund
\definecolor{maincolor}{HTML}{06798d}

% Display code listings
\usepackage{listings}
\usepackage{sourcecodepro}
\definecolor{mygreen}{rgb}{0,0.6,0}
\definecolor{mygray}{rgb}{0.5,0.5,0.5}
\definecolor{mymauve}{rgb}{0.58,0,0.82}
\definecolor{mylightgray}{gray}{0.98}

\lstset{ %
  backgroundcolor=\color{mylightgray},   % choose the background color; you must add \usepackage{color} or \usepackage{xcolor}; should come as last argument
  basicstyle=\footnotesize\ttfamily,        % the size of the fonts that are used for the code
  breakatwhitespace=false,         % sets if automatic breaks should only happen at whitespace
  breaklines=true,                 % sets automatic line breaking
  captionpos=b,                    % sets the caption-position to bottom
  commentstyle=\color{mygreen},    % comment style
  deletekeywords={...},            % if you want to delete keywords from the given language
  escapeinside={\%*}{*)},          % if you want to add LaTeX within your code
  extendedchars=true,              % lets you use non-ASCII characters; for 8-bits encodings only, does not work with UTF-8
  frame=none,	                   % adds a frame around the code
  keepspaces=false,                 % keeps spaces in text, useful for keeping indentation of code (possibly needs columns=flexible)
  keywordstyle=\color{blue},       % keyword style
  language=Octave,                 % the language of the code
  morekeywords={*,...},            % if you want to add more keywords to the set
  numbers=none,                    % where to put the line-numbers; possible values are (none, left, right)
  numbersep=5pt,                   % how far the line-numbers are from the code
  numberstyle=\tiny\color{mygray}, % the style that is used for the line-numbers
  rulecolor=\color{black},         % if not set, the frame-color may be changed on line-breaks within not-black text (e.g. comments (green here))
  showspaces=false,                % show spaces everywhere adding particular underscores; it overrides 'showstringspaces'
  showstringspaces=false,          % underline spaces within strings only
  showtabs=false,                  % show tabs within strings adding particular underscores
  stepnumber=2,                    % the step between two line-numbers. If it's 1, each line will be numbered
  stringstyle=\color{mymauve},     % string literal style
  tabsize=2,	                   % sets default tabsize to 2 spaces
  title=\lstname                   % show the filename of files included with \lstinputlisting; also try caption instead of title
}

% For title background image and logo on every page
\usepackage{wallpaper}

% set linespacing to onehalf globally
\usepackage[onehalfspacing]{setspace}

% set margins
\usepackage{geometry}
\geometry{
  left=1cm,
  right=1cm,
  top=2.12cm,
  bottom=2.18cm
}
% disable paragraph tab
\usepackage{blindtext}
\setlength{\parindent}{0pt}

% change caption color and size
\usepackage{caption}
\captionsetup{font=scriptsize,labelfont={color=gray,bf},textfont={color=gray}}

% used for hyperlinks
\usepackage[pdfborder={0 0 0}]{hyperref}

% used for tables
\usepackage{longtable}
\usepackage{booktabs}

% used to split long sequences
\usepackage{seqsplit}

\usepackage{graphicx}
\makeatletter
\def\maxwidth{\ifdim\Gin@nat@width>\linewidth\linewidth\else\Gin@nat@width\fi}
\def\maxheight{\ifdim\Gin@nat@height>\textheight\textheight\else\Gin@nat@height\fi}
\makeatother
% Scale images if necessary, so that they will not overflow the page
% margins by default, and it is still possible to overwrite the defaults
% using explicit options in \includegraphics[width, height, ...]{}
\setkeys{Gin}{width=\maxwidth,height=\maxheight,keepaspectratio}

\usepackage{float}
\floatplacement{figure}{H}



\usepackage{scrhack}
% Enables correct parsing of € symbol
\usepackage{eurosym}
\usepackage{tocstyle}
\usepackage{pict2e}
\usepackage{wasysym}
\usepackage{tikz}
\usepackage{calc}

% Use custom fonts (.ttf files need to be placed inside the fonts folder)
\usepackage{fontspec}
\setmainfont{BundesSans}[
  UprightFont={*-Regular},
  BoldFont={*-Bold}
]
\setsansfont{BundesSans}[
  UprightFont={*-Regular},
  BoldFont={*-Bold}
]
\setmonofont{BundesSans}[
  UprightFont={*-Regular},
  BoldFont={*-Bold}
]
\renewcommand*{\chapterheadstartvskip}{\vspace*{0.5cm}}
